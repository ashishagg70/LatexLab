\documentclass[12pt]{article}
\usepackage[utf8]{inputenc}
\usepackage[T1]{fontenc}
\usepackage{bera}
\usepackage{xspace}
\usepackage{amssymb}
\usepackage{listings}
\usepackage[nodayofweek]{datetime}
\usepackage{titling}
\renewcommand{\baselinestretch}{1}

\lstset{basicstyle=\ttfamily,escapeinside={||}}

\usepackage[letterpaper]{geometry}
\newcommand{\latex}{\LaTeX\xspace}

\setlength{\droptitle}{18em}

\title{\textbf{\huge Software Lab 4:\\TeX Lab}}
\author{\Huge Chill Assignment - Title Page}
\date{\emph{\today}}


\begin{document}

\maketitle 
\thispagestyle{empty}
\clearpage
\pagenumbering{arabic}
\tableofcontents \newpage

\section{About Me}\label{sec_aboutme}
\noindent
\textbf{!!! Yo Janta, Welcome to the CS699 Software Foundation Lab course !!!}\\
My name is Deepak Singh.You may be wondering how cs699 Assignments could be called as chilled , believe me guys this Lab will going to be chill. I am currently pursuing Mtech in Computer Science Department. This is a \latex document for the course \textbf{Software Lab} with course code CS699. I would like for this document to be typesetted perfectly which forces me to you use \latex. \latex uses various packages. I will elaborate about them in the following
subsections: \par
\textbf{Suggetion:-} Take assignments easy it will be fun.

\subsection{graphicx package}
This package is used to import tables, and figure in the document. Our doc-
ument type is article, and we are currently using 11pt font size, with a4 type
paper, which is specified in the beginning in .

\subsection{amssymb package}
This package is used to import mathematical symbols in the document. We
encapsulate the mathematical equations and symbols under \$, and they are
changed to maths symbols.

\section{Some History}\label{sec_somehistory}
I am ancient creature dwelling on this planet now referred to as "Earth". I
have been existing since the past 150393894.5 years. Do you see the use of a
package above in the number mention in the document. I have used something
to enunciate the numbers in a fashion such as a mathematical formulae.

Let us all try to replicate the text provided in this document.

\textit{P.S.: Please note that I am following the Section Title \textbf{Noun Capitaliza-
tion} in the document. This would be followed in the rest of the document,
henceforth.} \newpage

\section{Replication of this must be produced}\label{sec_rep}
\textbf{LaTeX} is a word processor and document markup language. It is dis-tinguished
from typical word processors such as Microsoft Word and Apple Pages in that
the writer uses plain text as opposed to format-ted text, relying on markup
tagging conventions to define the general structure of a document (such as
article, book, and letter), to stylise text throughout a document (such as \textbf{bold}
and \textit{italic}), and to add ci-tations and cross-referencing. A \textbf{TeX} distribution such
as \textbf{TeXlive} or \textbf{MikTeX} is used to produce an output file (such as PDF or DVI)
suitable for printing or digital distribution.\\

\textbf{LaTeX} is used for the communication and publication of scientific docu-
ments in many fields, including mathematics, physics, computer science, statis-
tics, economics, and political science. It also has a promi-nent role in the prepa-
ration and publication of books and articles that contain complex multilingual
materials, such as Sanskrit and Arabic. \textbf{LaTeX} uses the TeX typesetting pro-
gram for formatting its output, and is itself written in the TeX macro language.\\

\fontdimen3\font=0.6ex
\textbf{LaTeX} is widely used in academia. \textbf{LaTeX} can be used as a 
stan-dalone document preparation system, or as an intermediate format. In the latter role,
for example, it is often used as part of a pipeline for translating DocBook and 
other XML-based formats to PDF. The type-setting system offers programmable
desktop publishing features and extensive facilities for automating most as-
pects of typesetting and desk-top publishing, including numbering and cross-
referencing of tables and figures, chapter and section headings, the inclusion
of graphics, page layout, indexing and bibliographies.\\

Like \textbf{TeX, LaTeX} started as a writing tool for mathematicians and computer
scientists, but from early in its development it has also been taken up by schol-
ars who needed to write documents that include com-plex math expressions or
non-Latin scripts, such as Arabic, Sanskrit and Chinese.\\

\textbf{LaTeX} is intended to provide a high-level language that accesses the power
of \textbf{TeX}. \LaTeX\ comprises a collection of TeX macros and a program to process
\textbf{LaTeX} documents. Because the plain \textbf{TeX} formatting com-mands are elemen-
tary, it provides authors with ready-made commands\newpage for formatting and layout requirements such as chapter headings, foot-
notes, cross-references and bibliographies.

\fontdimen3\font=0.3ex
\textbf{LaTeX} was originally written in the early 1980s by Leslie Lamport at SRI In-
ternational. The current version is LaTeX2e. \textbf{LaTeX} is free soft- ware and is dis-
tributed under the \textbf{LaTeX} Project Public License (LPPL) \textbf{(Source Wikipedia)}.

\section{Opening and Compiling Tex Document}\label{sec_open}
First create a \textbf{.tex} file using text editor such as \textbf{Vi} or \textbf{Gedit} or \textbf{Kile}.

\section{Starting and Ending}\label{sec_startend}
A minimal input file looks like following

\vspace{1cm}
{\bfseries \textbackslash documentclass\{class\} \par 
\textbackslash begin\{document\}\par
\hspace{0.8cm} your text... \par
\textbackslash end \{document\}}\\
\noindent
where the class is a valid document class for \textbf{LaTeX}.
\subsection{Compiling the LaTeX Document}
We open the terminal and go to the directory in which our .tex file is stored
and the we execute the command\\

\textbf{pdflatex example.tex} \newpage

\section{Cross Reference}\label{sec_crossref}
One reason for numbering things like figures and equations is to refer the
reader to them, as in "Section 6 on Page 5 for more details".

I have referred to a section \ref{sec_aboutme}, and a page \pageref{sec_aboutme}

\subsection{\textbackslash label\{key\}}
A \textbf{\textbackslash label} command appearing in ordinary text assigns to key the number of the
current sectional unit; one appearing inside a numbered environ-ment assigns
that number to key.\\

\indent\indent
A key name can consist of any sequence of letters, digits, or punctu-
ation characters. Upper and lowercase letters are distinguished.

To avoid accidentally creating two labels with the same name, it is common
to use labels consisting of a prefix and a suffix separated by a colon or period.
Some conventionally-used prefixes:\\

\noindent
\textbf{ch} \hspace{0.5cm} for chapters\\
\textbf{sec} \hspace{0.5cm} for lower-level sectioning commands\\
\textbf{fig} \hspace{0.5cm} for figures\\
\textbf{tab} \hspace{0.5cm} for tables\\
\textbf{eq} \hspace{0.5cm} for equations\\

\section{Special Symbols}\label{sec_specsym}
Represent special symbols in \LaTeX\ \\

\noindent
\textbf{alpha \hspace{1cm} $\alpha$}\\
\textbf{phi \hspace{1cm} $\phi$}\\
\textbf{asterisk \hspace{0.8cm} $*$}\\
\newpage

\section{Section}\label{sec_section}
Sectioning commands provide the means to structure your text into units:

\begin{lstlisting}

|\textbf{\textbackslash part}|
            
|\textbf{\textbackslash chapter}|

|\textbf{(report and book class only)}|

|\textbf{\textbackslash section}|

|\textbf{\textbackslash subsection}|

|\textbf{\textbackslash subsubsection}|

|\textbf{\textbackslash paragraph}|

|\textbf{\textbackslash subparagraph}|
||
\end{lstlisting}

All sectioning commands take the same general form, e.g.,
\begin{lstlisting}
         |\textbf{\textbackslash chapter[toctitle]\{title\}}|
\end{lstlisting}
In addition to providing the heading title in the main text, the section title
can appear in two other places:
\begin{enumerate}
\item The table of contents.
\item The running head at the top of the page.
\end{enumerate} \par
You may not want the same text in these places as in the main text. To
handle this, the sectioning commands have an optional argument toctitle that,
when given, specifies the text for these other places.
\clearpage
\newgeometry{bottom=4cm}
Also, all sectioning commands have *-forms that print title as usual, butdo
not include a number and do not make an entry in the table of contents. \\

\setlength{\parskip}{0.2em}

For instance:

\begin{lstlisting}
            |\textbf{\textbackslash section*\{Preamble\}}|
\end{lstlisting}

\fontdimen3\font=0.5ex
The \textbf{\textbackslash appendix} command changes the way following sectional units are
numbered. The \textbf{\textbackslash appendix} command itself generates no text and does not affect the numbering of parts.\\

The normal use of this command is something like
\begin{lstlisting}

            |\textbf{\textbackslash chapter\{A Chapter\}}|
            |\textbf{...}|
            |\textbf{\textbackslash appendix}|
            |\textbf{\textbackslash chapter\{The First Appendix\}}|
            ||
            
\end{lstlisting}

\fontdimen3\font=2ex
The secnumdepth counter controls printing of section numbers. The setting suppresses heading numbers at any depth > level, where chapter is
level zero.

\begin{lstlisting}
            |\textbf{\textbackslash setcounter\{secnumdepth\}\{level\}}|
            ||
            ||
\end{lstlisting}

{\huge I think we have replicated the docu-
ment enough. Let us just concentrate
on learning features of the document
provided to us. We have successfully
demonstrated the the features such as
Sections, Subsections, Labelling, Bold,
Italics, Tabbing, Title Page, Huge, Large,
math symbols. Typing in a {\latex}document
to type in \textbf{LaTeX} code.\\}

Let us leave the rest of {\latex}for the out-lab.
\end{document}
